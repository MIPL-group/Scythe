%-----------------------------------------------------------------------------
%
%               Template for sigplanconf LaTeX Class
%
% Name:         sigplanconf-template.tex
%
% Purpose:      A template for sigplanconf.cls, which is a LaTeX 2e class
%               file for SIGPLAN conference proceedings.
%
% Guide:        Refer to "Author's Guide to the ACM SIGPLAN Class,"
%               sigplanconf-guide.pdf
%
% Author:       Paul C. Anagnostopoulos
%               Windfall Software
%               978 371-2316
%               paul@windfall.com
%
% Created:      15 February 2005
%
%-----------------------------------------------------------------------------


\documentclass[preprint]{sigplanconf}

% The following \documentclass options may be useful:

% preprint      Remove this option only once the paper is in final form.
% 10pt          To set in 10-point type instead of 9-point.
% 11pt          To set in 11-point type instead of 9-point.
% numbers       To obtain numeric citation style instead of author/year.

\usepackage{amsmath}

\newcommand{\code}[1]{\texttt{\footnotesize #1}}
\newcommand{\todo}[1]{{\bf \{TODO: {#1}\}}}

\def\modify#1#2#3{{\small\underline{\sf{#1}}:} {\color{red}{\small #2}}
{{\color{red}\mbox{$\Rightarrow$}}} {\color{blue}{#3}}}
%\renewcommand{\modify}[3]{{#3}}

\newcommand{\clmodify}[2]{\modify{Chenglong}{#1}{#2}}

\newcommand\mymargin[1]{\marginpar{{\flushleft\textsc\footnotesize {#1}}}}
\newcommand\yxmargin[1]{\mymargin{YX:\;#1}}
\newcommand\kcmargin[1]{\mymargin{KC:\;#1}}

\newcommand{\figref}[1]{Figure~\ref{#1}}
\newcommand{\eqnref}[1]{violation~\eqref{#1}}
\newcommand{\secref}[1]{Section~\ref{#1}}
\newcommand{\tblref}[1]{Table~\ref{#1}} 
\newcommand{\smalltitle}[1]{{\smallskip \noindent \bf  {#1}.\ }}
\newcommand{\smalltitlecolon}[1]{{\smallskip \noindent \bf  {#1}:\ }}
\newcommand{\smalltitleno}[1]{{\smallskip \noindent \bf  {#1}\ }}

\newcommand{\clmodifyok}[2]{#2}
\newcommand{\clmodifyno}[2]{#1}

\newcommand{\comment}[2]{{\small\color{magenta}\underline{\sf{#1}}:} {\color{magenta}{\small #2}}}



\newcommand{\cL}{{\cal L}}

\begin{document}

\special{papersize=8.5in,11in}
\setlength{\pdfpageheight}{\paperheight}
\setlength{\pdfpagewidth}{\paperwidth}

\conferenceinfo{CONF 'yy}{Month d--d, 20yy, City, ST, Country}
\copyrightyear{20yy}
\copyrightdata{978-1-nnnn-nnnn-n/yy/mm}
\copyrightdoi{nnnnnnn.nnnnnnn}

% Uncomment the publication rights you want to use.
%\publicationrights{transferred}
%\publicationrights{licensed}     % this is the default
%\publicationrights{author-pays}

%\titlebanner{banner}        % These are ignored unless
\preprintfooter{short description of paper}   % 'preprint' option specified.

\title{SQL Query Synthesis from Input Output Examples}
%\subtitle{Subtitle Text, if any}

\authorinfo{Name1}
           {Affiliation1}
           {Email1}
\authorinfo{Name2\and Name3}
           {Affiliation2/3}
           {Email2/3}

\maketitle

\begin{abstract}
This is the text of the abstract.
\end{abstract}

%\category{CR-number}{subcategory}{third-level}

% general terms are not compulsory anymore,
% you may leave them out
%\terms
%term1, term2

%\keywords
%keyword1, keyword2

\section{Introduction}

Relational database serves an important role in modern data management, and SQL is the most commonly used language in querying data from these database systems. However, though SQL is designed to be declarative, structurally complex queries remains highly challenging even for software developers: for example, in Stack Overflow, there is a tag ``greatest-n-per-group'' for writing SQL to solve ``argmax'' problems (querying the row with greatest or least values for each group), and there are more than 1,500 posts for this single problem. As for other non-expert end users including commodity traders, chemist, physicist, school administrators or even bank counters, the task querying the database system with SQL queries is even more challenging.

A traditional way to increase database usability is to design GUIs for different applications: developers predefine a set of parameterized queries, and users query data by filling a form using GUI, by which some query in the set will be instantiated and executed. This approach benefits users by hiding all details of underlying SQL queries, but on the other hand, its drawback is also obvious: the GUI are application specific and users can only query data using these predefined queries, which limits the ways for user to perform more complex query retrieval tasks. 

As a matter of fact, database designers are seeking for more user-friendly interfaces to reduce users' efforts in querying databases. Visual interfaces are designed by 

\appendix
\section{Appendix Title}

This is the text of the appendix, if you need one.

\acks

Acknowledgments, if needed.

% We recommend abbrvnat bibliography style.

\bibliographystyle{abbrvnat}

% The bibliography should be embedded for final submission.

\begin{thebibliography}{}
\softraggedright

\bibitem[Smith et~al.(2009)Smith, Jones]{smith02}
P. Q. Smith, and X. Y. Jones. ...reference text...

\end{thebibliography}


\end{document}
